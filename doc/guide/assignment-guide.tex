\documentclass[a4paper, 12]{article}
\usepackage{url}
\usepackage{pgf}

\setlength{\oddsidemargin}{0mm}
\setlength{\evensidemargin}{-14mm}
\setlength{\marginparwidth}{0cm}
\setlength{\marginparsep}{0cm}
\setlength{\topmargin}{2mm}
\setlength{\headheight}{0mm}
\setlength{\headsep}{0cm}
\setlength{\textheight}{240mm}
\setlength{\textwidth}{168mm}
\setlength{\topskip}{0mm}
\setlength{\footskip}{10mm}

% This is needed to handle unicode minus signs output by matplotlib pgf.
\usepackage[utf8]{inputenc}
\DeclareUnicodeCharacter{2212}{-}

\newcommand{\est}[1]{\expandafter\hat#1}
\newcommand{\code}[1]{\texttt{#1}}
\newcommand{\var}[1]{\sigma_{#1}^2}
\newcommand{\std}[1]{\sigma_{#1}}
\newcommand{\reffig}[1]{\mbox{Figure~\ref{fig:#1}}}


\title{ENMT482 Assignment 1 Guide}
\author{M.P. Hayes}
\date{}

\begin{document}
\maketitle


\section{Part A}

This part of the assignment is about localising the position of a
robot in 1-D using a number of sensors.  The goal is to create sensor
models, a motion model, and to use a Bayes filter to improve the
estimate.  The better the model, the better the estimation.


\subsection{Sensor models}

In general, for each sensor, you want a model of the form:
%
\begin{equation}
  Z = h(x) + V(x).
\end{equation}

  
\begin{enumerate}
\item Download the data files and example scripts from Learn.
  
\item Plot the sensor data using the file \url{calibration.csv} (see
  Python script \url{plot-calibration.py}). 

\item Choose the sensors you wish to fuse (ignore the hard one to
  start with).

 
\item Fit a parametric model to the data for each sensor using
  parametric identification given measured pairs of data $(x_n, z_n)$,
  see \reffig{fit}.  Let's say your model is:
  %
  \begin{equation}
    h(x) = a + b x + c x^2.
  \end{equation}
  %
  The goal is to find the parameters $a$, $b$, and $c$ that minimise
  the errors:
  %
    \begin{equation}
    v_n = z_n - h(x_n).
  \end{equation}

  If you are good at programming, there are many tools you can use for
  this.  The tricky aspect is dealing with outliers.  If you are not
  good at programming, tweak your model parameters using
  trial-and-error until you get a good fit.

  \begin{figure}[!h]
  \centering
  \input{figs/IRsensor-scatter.pgf}
  \caption{Non-linear sensor calibration data.}
  \label{fig:fit}    
  \end{figure}

  
  \item If you have a good model, the mean error is zero and the
    variance is minimized.  I suggest plotting the error, $v_n$, as a
    function of $x_n$ to see how good your model is.  For example, see
    \reffig{errors}.

    \begin{figure}[!h]
  \centering
  \input{figs/IRsensor-error.pgf}
  \caption{Error between measured data and model for non-linear sensor
    calibration data.}
  \label{fig:errors}  
  \end{figure}


    
\item If you have outliers, you should remove them from your dataset.
  The simple approach is to ignore values with large errors.  You may
  have to do this iteratively.

\item Determine how good each sensor is, i.e., what is its variance,
  see \reffig{histogram}.  The tricky aspect is that some sensors have
  a variance that varies with $x$.  If this is the case, you need to
  create a model for this, i.e., you need $\std{V(x)}$.  Note, if the
  variance changes slowly with $x$, a simple look-up table would
  suffice.

  \begin{figure}[!h]
  \centering
  \input{figs/IRsensor-error-histogram.pgf}
  \caption{Histogram of errors between measured data and model for
    non-linear sensor calibration data in the range $0 \le x \le 1$.}
  \label{fig:histogram}
  \end{figure}
  
\end{enumerate}

  
\subsection{Motion model}

Here you need a model of the form:
%  
\begin{equation}
  X_n = X_{n-1} + g(X_{n-1}, u_{n-1}) + W_n,
\end{equation}
%
where $u_{n-1}$ is the commanded speed.

\begin{enumerate}
\item Plot the commanded speed and the estimated speed (using the
  distance data) in the training files, \url{training1.csv} and
  \url{training2.csv}, and ponder what is going on.

\item Determine a motion model, $g(x_{n-1}, u_{n-1})$.  A crude model
  is simply
  %
  \begin{equation}
    g(x_{n-1}, u_{n-1}) = u_{n-1} \Delta t,
  \end{equation}
  %
  where $u_{n-1}$ is the current commanded speed.
  
  A good model will have a zero mean error and minimise the variance
  of the errors.  The process noise errors are given by
  %
  \begin{equation}
    w_n = x_n - x_{n-1} - g(x_{n-1}, u_{n-1}).
  \end{equation}

\item Determine the process noise variance, $\var{W}$, for your motion model.
  
\end{enumerate}


\subsection{Sensor fusion}


Here you need a Bayes filter; an extended Kalman filter is easiest to
start with.  However, you will need to linearise your non-linear
sensors around the current best estimate of the robot's position.

\begin{enumerate}
\item Predict the robot's position using the previous estimated
  position and your motion model.

\item Determine the variance of the predicted robot's position.  
  
\item For each sensor, invert its model $h(x)$ so that given a
  measurement $z$ you can have an estimate, $\est{x}$, for $x$.
  However, some non-linear models are not easily invertible and so you
  may need to use interpolation to find $\est{x}$ given $z$.

\item For each sensor, determine its noise variance $\var{V}(x)$ at
  the current best estimate for $x$.

\item For each sensor, convert the noise variance $\var{V}(x)$ into
  $\var{\est{X}}(x)$, where $\est{X}$ is the estimator for $X$.  This
  requires linearisation if your sensor model is non-linear (see the
  sensor fusion supplement in the lecture notes).

  \begin{figure}[!h]
  \centering
  \input{figs/IRsensor-linearised.pgf}
  \caption{Non-linear sensor model linearised around $x=5$.}
  \label{fig:linearised}
  \end{figure}
  
\item Combine the estimates from each sensor using a BLUE (see lecture
  notes).  To check that you have the correct weights, plot the
  weights for each sensor at each time-step.

\item Combine the prediction from the motion model with the estimates
  from the sensors using a BLUE.  This can be done at the same time as
  the previous step.

\item Determine the variance of the BLUE estimator (see lecture
  notes).

\item Rinse and repeat.
\end{enumerate}


\section{Part B}

This part of the assignment is about localising the position of a
robot in 2-D using fiducial markers (beacons) and a particle filter.
These markers are sensed by a camera on a Turtlebot2 robot to estimate
the local pose of the marker with respect to the robot.


The particle filter algorithm is written for you but you need to write
motion and sensor models.  If you are feeling clever, you might try
adapting the number of particles and/or not using knowledge of the
starting position.


\subsection{Motion model}

You can use either the velocity or odometry motion models.  The latter
has the advantage of decoupling the errors, see lecture notes.

To test your motion model, disable the sensor model (so that the
particle weights do not change) and see if the particles move by the
correct amount in the correct direction.  A useful Python script is
`test-motion-model.py`.

Once the particles are moving correctly, add some random behaviour to
the motion of each particle to mimic process noise.  The amount of
randomness depends on how good your motion model is.  However, it is
difficult to evaluate the process noise and I suggest that you tweak
this by trial and error.


\subsection{Sensor model}


To correctly implement the sensor model you will need to understand:
%
\begin{enumerate}
\item The difference between the robot and global (map) reference frames.

\item How to calculate the range and bearing of the beacons with
  respect to the robot given the estimated pose of the beacons.

\item How to calculate the range and bearing of the beacons with
  respect to each particle.

\item How to use the \code{arctan2} function.

\item How to determine the smallest angle between two vectors.  
\end{enumerate}
%
All these aspects are covered in the lecture notes.

Unfortunately, there is no calibration data for the fiducial marker
sensor.  I suggest modelling the sensor noise (in both range and
bearing) as Gaussian random noise and choosing the standard deviation
by trial and error.  Note, the standard deviation will vary with the
observed pose of the marker (it will be more accurate front-on than
side-on).

\subsection{Particle filter}

The more particles you have, the better the estimate but the slower
the computation.

Here's what I suggest you do:

\begin{enumerate}
\item Test motion model without adding noise to particles and disable
  sensor model.  The particles should move in the correct direction.

\item Test motion model with added noise and disable sensor model.
  The particles should move in the correct direction but spread out.

\item Enable sensor model but with large standard deviations for the
  range and bearing errors.  The particles should get closer together
  whenever a beacon is visible.

\item Reduce standard deviations in sensor model to get better
  tracking.
\end{enumerate}


\end{document}
