\documentclass[a4paper,12pt]{article}
\usepackage{color}
\usepackage{url}

\setlength{\oddsidemargin}{0mm}
\setlength{\evensidemargin}{-14mm}
\setlength{\marginparwidth}{0cm}
\setlength{\marginparsep}{0cm}
\setlength{\topmargin}{2mm}
\setlength{\headheight}{0mm}
\setlength{\headsep}{0cm}
\setlength{\textheight}{240mm}
\setlength{\textwidth}{168mm}
\setlength{\topskip}{0mm}
\setlength{\footskip}{10mm}
\setlength{\parindent}{8ex}

\newcommand{\comment}[1]{\emph{\color{blue}#1}}


\title{ENMT482 Assignment 1}
\author{Robbie T. Robot and Marvin P. Android, Group 42}
\date{}

\begin{document}
\maketitle

\section{Sensor fusion}


\subsection{Sensor models}

\comment{Explain your sensor models (calibration plots should go in
  the appendix).}


\subsection{Motion model}

\comment{Explain your motion model.  A figure of the estimated robot
  speed versus the speed predicted using your motion model would be
  useful.}


\subsection{Bayes filter}

\comment{Explain the Bayes filter you used.}


\subsection{Results}

\comment{Include plots of how close your estimate was to the true
  position, for the datasets with true position.  Include a plot of
  your estimate and its standard deviation for the test dataset.  If
  you use a Kalman filter, it would be useful to show the weights as a
  function of time.}


\subsection{Discussion}

\comment{Discuss what worked well and what improvements could be
  done.}


\section{Particle filter}

\subsection{Sensor model}

\comment{Explain your sensor model.}


\subsection{Motion model}

\comment{Explain your motion model.}


\subsection{Implementation}

\comment{Explain approach taken and number of particles used, etc.}


\subsection{Results}

\comment{Include plots of your estimated trajectory alongside the
  position from SLAM.}


\subsection{Discussion}

\comment{Discuss what you thought worked well about your estimation
  approach and what you could do to improve it.}


\section{SLAM}

\comment{Show the map you obtained from the Lab using the `gmapping`
  program and provide your observations regarding `gmapping`'s
  performance.}


%\appendix
%\section{Sensor models}

%\comment{Include your calibration plots here.}

\color{blue}
\section*{Instructions}

\begin{enumerate}
\item The reports can be created in Word or \LaTeX.  Use the
  appropriate template but delete all the instructions in blue.  

\item The page limit is five pages with an optional one page appendix.
  No title pages please.  We will deduct 10\% for every page over the
  page limit.  Do not squeeze the margin or use small fonts (12pt
  please).

\item Ensure your names and group number are in the title block.

\item No abstract, introduction, or conclusion is required.
  
\item Submit your reports as a PDF document through Learn.  We will
  deduct 10\% for non PDF documents.

\item Have a read of my guidelines for writing a report,
  \url{https://eng-git.canterbury.ac.nz/mph/report-guidelines/blob/master/report-guidelines.pdf}
  
\end{enumerate}


\end{document}

